% le caratteristiche richieste dall'università sono ellamport1986latexencate qui: https://stem.elearning.unipd.it/mod/book/view.php?id=234&chapterid=46#modalita
% 12pt: font richiesto dall'università
% twoside: i margini interni ed esterni sono scambiati per le pagine "a sinistra" e "a destra"
% openright: i capitoli cominciano in pagine dispari ("a destra")
% extreport: supporta 12pt
\documentclass[12pt,a4paper,twoside,openright]{extreport}

\usepackage{amsmath}                            % per avere più controllo sulle equazioni 
\usepackage{csquotes}                           % per le citazioni
\usepackage{enumitem}                           % per avere più controllo sulle enumerazioni
\usepackage[
    a4paper,
    top=2cm,bottom=2cm,
    outer=2cm,inner=3cm,
    includeheadfoot
]{geometry}                                     % margini richiesti dall'università
\usepackage{graphicx}                           % per le immagini
\usepackage{icomma}                             % per separare le cifre decimali con una virgola
\usepackage{minted}                             % per il codice con la colorazione della sintassi
\usepackage[a-1a]{pdfx}                         % formato richiesto dall'università
\usepackage[output-decimal-marker={,}]{siunitx} % per le unità di misura
\usepackage{subcaption}                         % per le sottodidascalie
\usepackage{svg}                                % per inserire file SVG

\usepackage[italian]{babel}
\selectlanguage{italian}

\usepackage[backend=bibtex,style=ieee]{biblatex}
\bibliography{bibliografia}

\usepackage{fontspec}
\setmainfont{Times New Roman} % carattere richiesto dall'università

\usepackage{setspace}
\onehalfspacing % interlinea richiesta dall'università

\sloppy % per evitare che il testo in \verb finisca oltre i margini

% questi valori vengono usati nella composizione del frontespizio
\title{Compressione di immagini tramite autoencoder, stato dell’arte e sviluppi futuri}
\author{Stella Filippo}
\date{GG/MM/AAAA}
\newcommand{\supervisor}{Prof. Cagnazzo Marco}
%\newcommand{\assistantsupervisor}{Cognome Nome}

\begin{document}
    \pagenumbering{roman}
    \pagestyle{empty} % per le prime pagine, non mostrare il numero di pagina

    \input{Capitoli/frontespizio}
    \cleardoublepage
    
    \vspace*{\stretch{1}}
\begin{flushright}
    \textit{Apes. Together. Strong.}
\end{flushright}
Testo di prova
\vspace{\stretch{4}}
    \cleardoublepage

    \pagestyle{plain} % comincia a mostrare il numero di pagina

    % NOTA: l'ambiente \abstract rimuove il numero della pagina e resetta il contatore delle pagine. 
    \chapter*{Sommario}
    Questo elaborato esplora il campo emergente della compressione delle immagini utilizzando autoencoder. L’obiettivo principale della ricerca è studiare e analizzare le tecniche esistenti per la compressione delle immagini tramite autoencoder.\\
    Nel primo capitolo, la tesi fornisce una panoramica dello stato dell’arte della compressione delle immagini, discutendo le tecniche tradizionali e come gli autoencoder si inseriscono in questo panorama. Viene fornita una spiegazione dettagliata del funzionamento degli autoencoder e dei loro vantaggi rispetto ai metodi di compressione delle immagini tradizionali.\\
    Il secondo capitolo si concentra sull’analisi delle tecniche esistenti per la compressione delle immagini tramite autoencoder. Vengono presentati vari studi di caso e vengono discusse le prestazioni di ciascuno in termini di qualità dell’immagine e rapporto di compressione.\\
    Nel terzo e ultimo capitolo, la tesi esplora i possibili sviluppi futuri nel campo della compressione delle immagini tramite autoencoder. Si discute di come le tecniche attuali potrebbero essere migliorate e si ipotizzano nuove direzioni di ricerca.\\
    In conclusione, questa tesi fornisce una panoramica completa dello stato dell’arte della compressione delle immagini tramite autoencoder e offre spunti preziosi per future ricerche in questo campo.\\

    \cleardoublepage

    \tableofcontents
    \cleardoublepage
    
    \listoffigures
    \cleardoublepage 

    \pagenumbering{arabic} % per assicurarsi che la numerazione araba cominci col primo capitolo 
    \setcounter{figure}{0} 
    
    \chapter*{Introduzione}
\addcontentsline{toc}{chapter}{Introduzione}
Nell’era moderna la crescente quantità di dati, di cui usufruiamo ogni giorno, sta ricevendo dagli esperti del settore molte attenzioni, In quanto il troughput e la quantità di memoria di cui disponiamo sui nostri dispositivi sono, seppur ad oggi ampiamente sufficienti, comunque limitati. Un’altra ragione di questa attenzione è la crescente diffusione di servizi in tempo reale che quindi richiedono di scambiare quantità di dati considerevoli in pochissimo tempo. Un esempio di tali servizi potrebbe essere l’uso della realtà aumentata in ambito medico o di ricerca. Comprimere i dati è quindi ormai una necessità, che si farà sempre più impellente con il crescere delle dimensioni dei dati che ci ritroveremo a dover scambiare.\\
A partire dai primi anni novanta infatti si sono iniziate a sviluppare alcune tecniche, a cui oggi si fa rifermento come tecniche tradizionali, per comprimere le immagini. Stiamo parlando ed esempio di JPEG e del suo successore JPEG2000, di più recente sviluppo sono invece i codec BPG e VVC.
Ultimamente l’attenzione dei ricercatori si è spostata su metodi che fanno uso dell'intelligenza artificiale. Questi metodi presentano diversi vantaggi rispetto ai metodi tradizionali, infatti molte volte permettono di ottenere performance migliori rispetto ai metodi classici.\\
In questo documento ci proponiamo di fornire una panoramica dei metodi di compressione tradizionali più usati, e di quelli che fanno uso dell'intelligenza artificiale che hanno fornito un maggiore contributo allo sviluppo di questi ultimi. Dopo aver presentato le varie tecniche vogliamo fornire una valutazione delle prestazioni in modo da poterli comparare ed evidenziare potenzialità e difetti di ognuno.\\
Per concludere vorremmo fornire alcuni spunti che potrebbero favorire l’adozione di queste tecniche più recenti nella nostra vita digitale di tutti i giorni o suggerire nuovi possibili fronti di ricerca.\\
Iniziamo fornendo una spiegazione di cosa sia la compressione. La compressione è un processo che mira a minimizzare il numero di bit utilizzati per rappresentare una certa informazione senza intaccarne drasticamente la qualità. Questo obbiettivo viene raggiunto riducendo le ridondanze ed eliminando i dati non necessari.
Tutti i framework di compressione consistono in una coppia codificatore-decodificatore. Data un’immagine da comprimere $x$ il codificatore, composto da una trasformata $\epsilon$ e una funzione di quantizzazione $Q$, può essere espresso con la seguente formulazione.\\
\begin{equation}\label{eq:eqCondificatore}
    y = Q(\epsilon (x;\theta_{\epsilon}))
\end{equation}
Dove $\theta_{\epsilon}$ denota i parametri del codificatore.
Per riottenere la rappresentazione dell’immagine il decodificatore ricostruisce l’immagine $\hat{x}$ dal codice $y$ nel seguente modo.\\
\begin{equation}\label{eq:eqDecodificatore}
    \hat{y} = D(y;\theta_{D}) = D(Q(\epsilon (x;\theta_{\epsilon}));\theta_{D})
\end{equation}
Dove $\theta_{D}$ denota i parametri del decodificatore e $D$ è una funzione che tenta di ricostruire l'immagine originale.\cite{hu2021learning}\\
Tutti gli algoritmi di compressione possono essere raggruppati in due macro categorie. Gli algoritmi lossy o con perdita e quelli lossless o senza perdita. Come si può già intuire dal nome gli algoritmi lossless comprimono le informazioni senza scartare informazioni, si limitano quindi ad applicare delle trasformate per ottenere delle nuove rappresentazioni più efficienti. Gli algoritmi lossy invece ammettono la possibilità di scartare delle informazioni superflue che non vanno ad intaccare drasticamente la qualità percepita dall'utilizzatore, in modo da poter comprimere ulteriormente.\\
Se andiamo ad osservare gli algoritmi di codifica lossy possiamo scomporli tutti in almeno tre blocchi principali.\cite{sadeeq2021image} \\
Un primo blocco si occupa di convertire l’immagine in una rappresentazione latente, tramite l’applicazione di una trasformata, in un altro dominio che permette di rappresentare l’informazione da comprimere in modo più sparso.\\
Successivamente si ha un blocco che si occupa di effettuare la quantizzazione, ovvero di mappare i valori in ingresso in un insieme di valori finito e di dimensione più piccola rispetto a quello di ingresso. In questo passaggio si realizza la perdita di informazione caratteristica della codifica lossy.\\
Un ultimo blocco si occupa di effettuare la codifica entropica, in modo da comprimere ulteriormente l’informazione mappando i simboli usati più spesso con pochi bit.\\
Questo schema a blocchi è rappresentato nelli'mmagine \ref{fig:LossyCompressorDiagram}\\
\begin{figure}[t!]
    \centering
    \includegraphics[width=0.8\textwidth]{Immagini/LossyCompressorDiagram.png}
    \caption{Diagramma di compressione Lossy}
    \label{fig:LossyCompressorDiagram}
\end{figure}

    \chapter{Metodi tradizionali}
Se andiamo ad osservare gli algoritmi di codifica Lossy possiamo scomporli tutti in almeno tre blocchi principali. \cite{sadeeq2021image} \\
Un primo blocco si occupa di convertire l’immagine in una rappresentazione latente, tramite l’applicazione di una trasformata, in un altro dominio che permette di rappresentare l’informazione da comprimere in modo più sparso.\\
Successivamente si ha un blocco che si occupa di quantizzazione, ovvero di mappare i valori in ingresso in un insieme finito di dimensione più piccola rispetto a quello di ingresso. In questo passaggio so realizza la perdita di informazioni caratteristica della codifica Lossy.\\
Un ultimo blocco si occupa di effettuare la codifica entropica, in modo da comprimere ulteriormente l’informazione mappando i simboli usati più spesso con pochi bit.

\begin{figure}[ht]
    \centering
    \includegraphics[width=0.8\textwidth]{Immagini/LossyCompressorDiagram.png}
    \caption{Diagramma di compressione Lossy}
    \label{fig:LossyCompressorDiagram}
\end{figure}

Gli algoritmi di compressione tradizionali usano trasformate statiche all’interno del primo blocco, messe a punto in numerosi anni di ricerca. Questa staticità non permette a questi metodi di adattarsi dinamicamente a tutti i tipi di contenuti delle immagini. \cite{cheng2018deep}  Inoltre rende il processo di sviluppo di un nuovo algoritmo di compressione un processo lungo che richiede anni di studi e progettazione. \cite{ cheng2018deep}
Andiamo ora a parlare brevemente dei metodi che andremo a considerare quando valuteremo le prestazioni dei vari algoritmi per poterli confrontare.

\section{JPEG}
Questo metodo di compressione sviluppato nel 1992 è diventato in poco tempo ed è attualmente il formato di compressione più diffuso. Nonostante in numerosi tentativi di sostituirlo con formati più moderni, questo è rimasto ancora ad oggi il formato più usato. Il JPEG si basa sull’utilizzo della DCT per realizzare la rappresentazione sparsificata dell’immagine originale. \cite{ 125072} \\
IMMAGINE CON JPEG

\section{JPEG2000}
Nel 2001 con la crescente diffusione di internet e con l’aumento di dimensione delle immagini e la richiesta di una maggiore qualità da parte degli utenti viene sviluppato questo nuovo formato chiamato appunto JPEG2000 per l’anno in cui è stato sviluppato.\\
JPEG2000 non utilizza la DCT come il suo predecessore ma vien introdotta una nuova trasformata la DWT o Discrete Wavelet Transformat che si propone di meglio identificare e comprimere i bordi delle figure che compongono le immagini. JPEG2000 quindi voleva essere un formato di qualità superiore con una compressione più efficiente. \cite{952804}
IMMAGINE CON JPEG2000

\section{BPG}
Questo formato è stato sviluppato da Fabrice Bellard nel 2014 come sostituto all’ormai affermato formato JPEG. Questo metodo di codifica si basa sulla codifica intra-frame del codec HEVC o H.265. \cite{BPGImageformat} Bellard voleva realizzare un formato molto leggero che potesse fornire immagini più compresse rispetto a JPEG, ma con una qualità superiore.\\
IMMAGINE CON BPG

\section{VVC}
Versatile Video Coding (VVC) o H.266 è lo standard di codifica video più recente, finalizzato nel luglio 2020. È stato sviluppato dal Joint Video Experts Team (JVET) dell'ITU-T Video Coding Experts Group (VCEG) e dell'ISO/IEC Moving Picture Experts Group (MPEG) per soddisfare la crescente richiesta di una migliore compressione video e per supportare una più ampia gamma di contenuti multimediali attuali e applicazioni emergenti come contenuti HDR, a 360°, per la Realtà Virtuale (VR) o la Realtà Aumentata (AR).\cite{9503377}\\
IMMAGINE CON VVC

    \chapter{Metodi con apprendimento automatico}

\section{Ballé 2018}

\section{Cheng 2020}

\section{Yang 2021}

\section{Wang 2022}



    \chapter{Valutazione delle prestazioni}
In questo capitolo vogliamo comparare gli algoritmi tradizionali descritti nel capitolo 1 con due degli algoritmi descritti nel capitolo 2, nello specifico Ballè et al. \cite{minnen2018joint} e Cheng et al. \cite{cheng2020learned}. Purtroppo non siamo stati in grado di testare anche Wang et al. \cite{wang2022neural} in quanto, nonostante abbiano reso il modello disponibile, non ne hanno fornito una versione pre addestrata.\\
L’hardware su cui sono stati eseguiti i test è un computer dotato di processore AMD Ryzen 5 5600X con 6 core e 12 thread a 3.7GHz, munito di 16GB di ram DDR4 a 3200MHz e come sistema operativo una distribuzione Fedora Linux 38 con kernel linux 6.5.7-200. Vogliamo sottolineare anche che per realizzare delle misurazioni più vicine alla realtà possibili abbiamo deciso di non utilizzare GPU, in quanto non tutti i computer sono dotati di GPU su cui è possibile eseguire Tensorflow o PyTorch e non abbiamo preso particolari precauzioni per garantire l’esecuzione esclusiva del codice, abbiamo lasciato che il codice concorresse con il sistema operativo e le applicazioni in background per l’allocazione del processore, tutto per simulare un ambiente più simile ad un caso reale.\\
Per effettuare i vari test sono stati realizzati dei notebook in python 3.11 su jupyterlab 4.0.7. I codificatori che abbiamo utilizzato sono i seguenti. Per comprimere le immagini con JPEG abbiamo utilizzato Pillow 10.0.1 \cite{PillowLibrary} sviluppato da Jeffrey A. Clark. Per comprimere le immagini con JPEG2000 abbiamo utilizzato OpenJPEG 2.5.0 \cite{OpenJPEGLibrary} sviluppato dall’ Université de Louvain. Per comprimere con BPG abbiamo usato il framework messo a disposizione da F.Bellard \cite{BPGImageformat} nella versione 0.9.8. Per comprimere con VVC abbiamo usato il framework distribuito da Fraunhofer HHI \cite{VVCLibrary} che comprende l’encoder vvencapp 1.9.1 e il decocder vvdecapp 2.1.2. Per comprimere con le due reti abbiamo utilizzato la libreria compressai 1.2.4 \cite{CompressAILibrary} in cui sono presenti le implementazioni pre allenate dei due modelli che ci interessano. Dei due modelli sono presenti due versioni, per Ballé et al. \cite{minnen2018joint} è fornita un implementazione in cui la media della distribuzione gaussiana viene bloccata a 0 e una in cui la media è un parametro determinato durante la compressione, per Cheng at al. 2020 \cite{cheng2020learned} viene fornita una versione che fa uso degli attention module e una in cui sono disabilitati.\\
Per valutare i vari metodi abbiamo utilizzato 24 immagini non compresse di dimensione $2048\:x\:3072$ con spazio di colore RGB a 8 bit per canale, prese dal database di Kodak \cite{KodakDataset}. I vari algoritmi sono stati eseguiti più volte con vari livelli di qualità, che andremo ad approfondire successivamente, ed infine sono state calcolate le metriche sui risultati ottenuti.\\

\section{Metriche utlizzate}
Per valutare le prestazioni facciamo affidamento ad alcune metriche che ci permettono di confrontare le varie tecniche. Molte di queste metriche sono oggettive, alcune invece cercano di quantificare il più fedelmente possibile quella che è la qualità percepita da un osservatore umano.
Passiamo ora ad introdurre brevemente le metriche e come vengono calcolate.\\

\subsection{BPP}
Per valutare quanto un’immagine sia stata compressa utilizziamo il numero di bit necessari per rappresentare un pixel dell’immagine $x$, o bit per pixel. Il calcolo \ref{eq:bpp} di questo parametro è molto semplice ed intuitivo.\\
\begin{equation}\label{eq:bpp}
    bpp(x) = \dfrac{taglia (x)}{larghezza (x) \cdot altezza (x)}
\end{equation}\\
Dove con taglia indichiamo il numero di bit occupati dall’immagine compressa, con larghezza intendiamo il numero di pixel in una riga orizzontale dell’immagine e con altezza intendiamo il numero di pixel in una riga verticale dell’immagine.\\
Lo spezzone di codice \ref{code:bppComputation} mostra come è stato calcolata la metrica bit per pixel.\\
\begin{adjustbox}{max width=\textwidth}
    \begin{lstlisting}[language=Python, caption=Spezzone di codice per il calcolo dei bit per pixel, label=code:bppComputation]
        from PIL import Image
        import os
        
        image = Image.open(file)
        file_size = os.path.getsize(file) * 8
        pixels = image.width * image.height
        bits_per_pixel = file_size/pixels
    \end{lstlisting}
\end{adjustbox}

\subsection{Tempo di codifica}
Per valutare la velocità con la quale un algoritmo di codifica riesce a produrre la rappresentazione compressa di un’immagine andiamo a misurare il tempo che intercorre tra la chiamata al codificatore e il termine dell’esecuzione del processo di codifica. Avendo utilizzato python per richiamare i codificatori abbiamo usato il modulo timeit di per calcolare i tempi di esecuzione.\\
Lo spezzone di codice \ref{code:CompressionTime} mostra come è stata usata la libreria per calcolare i tempi di esecuzione.\\
\begin{adjustbox}{max width=\textwidth}
    \begin{lstlisting}[language=Python, caption=Spezzone di codice per il calcolo del tempo di compressione, label=code:CompressionTime]
        import timeit
        
        starttime = timeit.default_timer()
        call_to_encoder() #Chiamata encoder
        endtime = timeit.default_timer()
        execution_time = endtime-starttime #Caloclo del tempo in secondi
    \end{lstlisting}
\end{adjustbox}

    
\subsection{PSNR}
L’ultima delle metriche oggettive che andiamo a considerare è il Peak Signal to Noise Ratio o PSNR. Questa metrica, espressa in scala logaritmica, rappresenta il rapporto tra la potenza del segnale originale e la potenza del rumore introdotto dal processo di compressione, dunque più alto è il valore del PSNR più l’immagine compressa è fedele all’originale.\\
Per poter calcolare il PSNR occore definire cosa sia e come si calcola l’MSE per immagini a colori, ovvero con più componenti.\\
L’ Errore Quadratico Medio o MSE indica la distanza al quadrato tra il valore di un pixel dell’immagine e il valore dello stesso pixel nell’immagine distorta. Per calcolare l’MSE di una singola componente di colore si utilizza la formula \ref{eq:MSE}, dove $M$ ed $N$ indicano rispettivamente la larghezza e l'altezza in pixel delle immagini orginale $R$ e compressa $C$.\\
\begin{equation}\label{eq:MSE}
    MSE(R,C) = \dfrac{1}{MN} \sum_{i=0}^{M-1} \sum_{j=0}^{N-1} || R(i,j) - C(i,j) ||^{2}
\end{equation}\\
Nel caso di immagini a colori però non abbiamo una sola componente ma ne abbiamo al minimo due, bisogna dunque definire come combinare gli MSE delle singole componenti, in base allo spazio di colore scelto, in un’unica metrica globale per l’immagine.\\
Nel nostro caso, avendo scelto lo spazio di colore YUV, la combinazione degli MSE delle singole componenti si ottiene con la formula \ref{eq:wMSE}, nella quale possiamo osservare che l’MSE globale dell’immagine non è altro che la somma pesata degli MSE delle singole componenti.\\
Questi pesi sono utilizzati per attribuire più importanza al canale $Y$, in quanto le informazioni in questo canale sono quelle maggiormente responsabili della qualità dell’immagine.\\
\begin{equation}\label{eq:wMSE}
    MSE(R,C) = (\dfrac{3}{4}) \cdot MSE_{Y}(R,C) + (\dfrac{1}{8}) \cdot MSE_{U}(R,C)  +  (\dfrac{1}{8}) \cdot MSE_{V}(R,C)
\end{equation}\\
Avendo definito come si calcola l’MSE pesato, il calcolo del PSNR pesato si ottiene con la formula \ref{eq:PSNR}, in cui $I$ indica il massimo valore che il pixel di un canale può assumere, nel nostro caso essendo i canali nello spazio di colore YUV rappresentati con 8 bit ciascuno, il valore di $I$ è 255.\\
\begin{equation}\label{eq:PSNR}
    PSNR(R,C)  =  10 * \log{\dfrac{I^2}{MSE(R,C)}}_{10}
\end{equation}\\
Questa metrica è uno standard nella comunità scientifica per l’analisi della qualità dei sistemi che operano con immagini, solitamente viene calcolata per immagini con gli spazio di colore YUV o YCbCr.\\
Per il calcolo abbiamo usato la libreria scikit-image 0.22.0 per calcolare l’MSE dei singoli canali per poi combinarli e calcolare il PSNR, come possiamo vedere nello spezzone di codice \ref{code:PSNRComputation}.\\
\begin{adjustbox}{max width=\textwidth}
    \begin{lstlisting}[language=Python, caption=Spezzone di codice per il calcolo del PSNR pesato, label=code:PSNRComputation]
        from skimage import metrics
        
        YMSE = metrics.mean_squared_error(OY,Y)
        UMSE = metrics.mean_squared_error(OU,U)
        VMSE = metrics.mean_squared_error(OV,V)
        MSE = (3/4)*YMSE + (1/8)*UMSE + (1/8)*VMSE
        psnr = 10 * numpy.log10((255*255) / MSE)
        print(‘PSNR: ’+str(msssim)+’dB’)
    \end{lstlisting}
\end{adjustbox}   
    

\subsection{MSSIM}
La prima metrica che cerca di fornire un indice di qualità percepita che andremo a considerare è il MultiScale Structural Similarity for IMage quality assessment o MS-SSIM. \cite{wang2003multiscale}.\\
Lo sviluppo di questo indice si basa sull’assunto che il sistema visivo umano è altamente adatto per estrarre informazioni strutturali dalle immagini, dunque una misura di similarità strutturale dovrebbe fornire una buona approssimazione della qualità percepita da un osservatore umano.\\
Per il calcolo dell’MS-SSIM, come possiamo vedere nell’equazione \ref{eq:ms-ssim}, si deve eseguire il prodotto pesato di tre indici, un indice di luminanza $l$, un indice di contrasto $c$ e un indice di struttura $s$, per la stessa immagine scalata $M$ volte. Questi indici sono pesati da tre esponenti, rispettivamente $\alpha_{M}\:\beta_{j}\:\gamma_{j}$, e come possiamo osservare solamente due dei componenti vengono pesati per ogni scalatura dell’immagine, la luminanza invece viene calcolata solo per l’indice di qualità $M$.\\
\begin{equation}\label{eq:ms-ssim}
    MS-SSIM(R,C) = [l_{M}(R,C)]^{\alpha_{M}} \cdot \prod_{j=1}^{M}[c_{j}(R,C)]^{\beta_{j}} [s_{j}(R,C)]^{\gamma_{j}} 
\end{equation}\\
L’indice $M$ rappresenta la scalatura delle immagini in ingresso, se l’indice è uguale a 1 ci stiamo riferendo esattamente alle immagini da valutare. Ad ogni incremento di $M$ all’immagine originale viene applicato un filtro passa basso e viene sotto campionata di un fattore 2. Dopo esattamente $M-1$ iterazioni l’immagine non sarà più riducibile e la computazione termina.\\
Questa è la seconda metrica standard che viene utilizzata dalla comunità scientifica per valutare le prestazioni di algoritmi che operano su immagini, in quanto a differenza del PSNR, questa rappresenta un giudizio più soggettivo.\\
Per il calcolo dell’MS-SSIM abbiamo usato la libreria pytorch-msssim 1.0.0 che fornisce una semplice funzione per il calcolo di questa metrica, l’unica accortezza da dover prendere è quella di convertire le immagini da valutare in tensori normalizzati prima di passare le immagini alla funzione, come possiamo vedere nello spezzone di codice \ref{code:MSSSIMComputation}.\\
\begin{adjustbox}{max width=\textwidth}   
    \begin{lstlisting}[language=Python, caption=Spezzone di codice per il calcolo dell'MS-SSIM, label=code:MSSSIMComputation]
        import torch
        from skimage import metrics
        from torchvision import transforms
        
        device = 'cuda' if torch.cuda.is_available() else 'cpu'
        
        reference = transforms.ToTensor()(reference_image).unsqueeze(0).to(device)
        compressed = transforms.ToTensor()(compressed_image).unsqueeze(0).to(device)
        msssim = ms_ssim(reference, compressed, data_range=1, size_average=True)
        print(‘MS-SSIM: ’+str(msssim))
    \end{lstlisting}
\end{adjustbox} 


\subsection{LPIPS}
L’ultima metrica che andremo ad utilizzare è anch'essa soggettiva e utilizza delle reti neurali per valutare la distanza percepita tra l’immagine originale e l’immagine compressa, la tecnica proposta da Zhang et al. nel 2018 \cite{zhang2018unreasonable} prende il nome di Learned Perceptual Image Patch Similarity o LPIPS.\\
L’osservazione su cui si basa lo sviluppo di questa metrica è il fatto che l’attivazione interna dei neuroni di reti utilizzata per compiti di classificazione di immagini ad alto livello possono essere utilizzate per calcolare una distanza percepita tra due immagini. Durante lo sviluppo di questa metrica il team di Zhang et al. ha scoperto non solo che questa si rivela essere un’ottima metrica, ma riesce anche ad emulare molto bene i giudizi dati da degli osservatori umani, la maggior parte delle volte anche in modo migliore rispetto ad altre metriche più affermate, come il precedentemente citato MS-SSIM.\\
Le reti che sono state valutate dal team di Zhang et al. sono SqueezeNet, AlexNet e VGG, a queste reti vengono fornite le due immagini $x$ e $x_{0}$, rispettivamente originale e distorta. Dalle reti vengono poi estratte le feature da $L$ livelli, rispettivamente $\hat{y}^l$ e $\hat{y}_{0}^l$ e vengono normalizzate rispetto alla dimensione dei canali. Le attivazioni dei vari canali vengono poi pesate con un vettore di pesi $w^l$ e ne viene calcolata la distanza $L_{2}$, come possiamo vedere nell’equazione \ref{eq:LPIPSExtraction}.\\
\begin{equation}\label{eq:LPIPSExtraction}
    d(x,x_{0}) = \sum_{l}\dfrac{1}{H_{l}W_{l}} \sum_{h,w} || w_{l} \cdot (\hat{y}_{hw}^{l} - \hat{y}_{0,hw}^{l}) ||_{2}^{2}
\end{equation}\\
Noi abbiamo scelto di utilizzare la metrica LPIPS con rete AlexNet, in quanto dai risultati sperimentali, nonostante non sia la tecnica più avanzata, fornisce la valutazione più simile a quella data da degli osservatori umani.\\
Il team di Zhang et al. fornisce anche una libreria per python che abbiamo utilizzato per i nostri test nella versione 0.1.4, nello spezzone di codice \ref{code:LPIPSComputation} mostriamo l'uso di tale libreria.\\
\begin{adjustbox}{max width=\textwidth}   
    \begin{lstlisting}[language=Python, caption=Spezzone di codice per il calcolo di LPIPS con AlexNet, label=code:LPIPSComputation]
        import torch
        from torchvision import transforms
        import lpips
        
        loss_fn_alex = lpips.LPIPS(net='alex')
        
        reference = transforms.ToTensor()(reference_image).unsqueeze(0).to(device)
        compressed = transforms.ToTensor()(compressed_image).unsqueeze(0).to(device)
        d = loss_fn_alex(transforms.Normalize(mean=(0.5, 0.5, 0.5),
                         std=(0.5, 0.5, 0.5))(reference),transforms.Normalize(mean=(0.5, 0.5, 0.5),
                         std=(0.5, 0.5, 0.5))(compressed))[0].item()
        print(‘LPIPS: ’+str(d))
    \end{lstlisting}
\end{adjustbox} 
    

\section{Presentazione dei risultati}
Andiamo ora a presentare i risultati sperimentali ottenuti dalla compressione delle immagini del dataset Kodak con gli algoritmi precedentemente descritti. Per ogni metodo di compressione vengono analizzati cinque diversi livelli di qualità comparati con le metriche appena descritte. I livelli vengono scelti andando ad agire sui parametri di qualità degli encoder, per cercare di mantenere il confronto equo abbiamo cercato di ottenere circa gli stessi livelli per ogni metodo di compressione. Tutte i valori che andremo a presentare sono le medie delle metriche sulle 24 immagini del dataset.\\
Il primo livello significativo si trova in corrispondenza di $0.16bpp$, i tempi di compressione per questo livello di qualità sono visibili nell’immagine \ref{fig:times16}. Il secondo livello si trova in corrispondenza di $0.21bpp$, dei cui tempi di compressione sono riportati nell’immagine \ref{fig:times21}. Il terzo livello che andiamo a considerare si ha per $0.34bpp$, i tempi di compressione per questo livello sono riportati nel grafico\ref{fig:times34}.\\
Un ulteriore livello di qualità si ha in corrispondenza di $0.07bpp$, JPEG non riesce però a comprimere fino a questi livelli, mentre la misurazione di JPEG2000 non è stata fatta perché si è preferito prendere una misura per un livello di qualità più elevato. I tempi di compressione per questo livello sono stati comunque calcolati in quanto è interessante osservare il comportamento del codec VVC, e sono riportati nell’immagine \ref{fig:times07}.\\
L’ultimo livello di qualità invece si ha per $0.41bpp$ dove però non abbiamo le misurazioni delle reti, in quanto non sono presenti i modelli pre addestrati per queste qualità. I tempi di compressione per quest’ultimo livello sono visibili nella figura \ref{fig:times41}.\\

\begin{figure}[!h]
    \centering
    \includegraphics[width=1\textwidth]{Immagini/METRICS/times@0.07bpp.png}
    \caption{Tempi di compressione a 0.07 bpp}
    \label{fig:times07}
\end{figure}
\newpage
\begin{figure}[!h]
    \centering
    \includegraphics[width=1\textwidth]{Immagini/METRICS/times@0.16bpp.png}
    \caption{Tempi di compressione a 0.16 bpp}
    \label{fig:times16}
\end{figure}
\hspace{0.5cm}
\begin{figure}[!h]
    \centering
    \includegraphics[width=1\textwidth]{Immagini/METRICS/times@0.21bpp.png}
    \caption{Tempi di compressione a 0.21 bpp}
    \label{fig:times21}
\end{figure}
\newpage
\begin{figure}[!h]
    \centering
    \includegraphics[width=1\textwidth]{Immagini/METRICS/times@0.34bpp.png}
    \caption{Tempi di compressione a 0.34 bpp}
    \label{fig:times34}
\end{figure}
\hspace{0.5cm}
\begin{figure}[!h]
    \centering
    \includegraphics[width=1\textwidth]{Immagini/METRICS/times@0.41bpp.png}
    \caption{Tempi di compressione a 0.41 bpp}
    \label{fig:times41}
\end{figure}
\clearpage 
Come possiamo osservare nei grafici i tempi di compressione delle reti aumentano all’aumentare della qualità, i tempi di compressione dei metodi tradizionali invece non variano sensibilmente all’aumentare della qualità. Il codec VVC costituisce però un’eccezione a questo comportamento costante dei metodi tradizionali, in quanto possiamo vedere che il tempo aumenta all’aumentare della qualità. Questo comportamento si ha per la complessità di H.266, in quanto deve trovare il miglior metodo di compressione tra quelli a sua disposizione.\\
Passiamo ora a presentare le metriche di qualità dell’immagine, partiamo dal grafico del PSNR \ref{fig:PSNRGraph}, come possiamo vedere nel grafico, la migliore qualità di compressione si ha con il codec VVC, seguito poi da Cheng 2020, Ballé 2018, BPG, JPEG 2000 ed infine JPEG. H.266 si conferma quindi l’attuale stato dell’arte per la compressione di immagini osservando il grafico del PSNR, che ricordiamo essere una metrica oggettiva.\\
\begin{figure}[!h]
    \centering
    \includegraphics[width=0.9\textwidth]{Immagini/METRICS/PSNR.png}
    \caption{Grafico PSNR}
    \label{fig:PSNRGraph}
\end{figure}\\
Il prossimo grafico che andremo a commentare è il grafico dell’MS-SSIM \ref{fig:MSSSIMGraph}, in questo grafico possiamo osservare come JPEG, JPEG 2000 e BPG si confermino i metodi che forniscono le qualità di compressione peggiori. Diversamente da quanto osservato nel grafico del PSNR, secondo questo grafico il miglior metodo di compressione è Cheng2020, seguito da Balle2018 e come terzo abbiamo VVC.\\
Ricordiamo che l’MS-SSIM è invece una metrica che cerca di replicare la valutazione del sistema visivo umano.\\
\begin{figure}[!h]
    \centering
    \includegraphics[width=0.9\textwidth]{Immagini/METRICS/MS-SSIM.png}
    \caption{Grafico MS-SSIM}
    \label{fig:MSSSIMGraph}
\end{figure}\\
L’ultima metrica è invece la più recente, andiamo ora a presentare i risultati presenti nel grafico della metrica LPIPS con AlexNet \ref{fig:LPIPSGraph}. Come possiamo vedere in questo grafico VVC si conferma il metodo migliore, seguito da Cheng2020, Ballé2018, BPG, JPEG 2000. Per quanto riguarda JPEG otteniamo un comportamento inaspettato in quanto i primi due punti sono in linea con le aspettative, il successivo invece è vicino a JPEG200 e gli ultimi due lo superano addirittura.\\
Questo comportamento ci ha sorpreso in quanto ci aspettavamo che la qualità percepita fosse vicina a quella di JPEG 2000 ma non che la superasse.\\
\begin{figure}[!h]
    \centering
    \includegraphics[width=0.9\textwidth]{Immagini/METRICS/LPIPS.png}
    \caption{Grafico LPIPS con AlexNet}
    \label{fig:LPIPSGraph}
\end{figure}\\
Dopo aver presentato e commentato questi grafici ci sentiamo di affermare che le reti da noi valutate svolgono un lavoro molto buono, che supera i metodi di compressione tradizionali più usati. Non riescono però ancora a superare il codec VVC che attualmente rimane lo stato dell’arte per la compressione di immagini.\\

    \chapter{Sviluppi futuri}

\section{Possibile utilizzo di SADAM}

In una ricerca del 2018 Ballé introduce Spectral ADAM (SADAM) \cite{balle2018efficient}, un'alternativa all’algoritmo ADAM \cite{kingma2014adam}, per l’addestramento di reti di compressione.\\
Se consideriamo una funzione di perdita $L$ che consiste in una somma di funzioni di una proiezione lineare dei vettori in input $x$ \ref{eq:SGDLoss}, dove $H$ è una matrice di filtri $h_{i}$. I filtri sono una parte dei parametri che devono essere ottimizzati in fase di addestramento, per ogni filtro la regola di aggiornamento del gradient descent \ref{eq:SGDUpdate} sottrae il gradiente della funzione di perdita moltiplicato per lo step size $\rho$.\\
\begin{equation}\label{eq:SGDLoss}
L = \sum_{x} l(x) \:\: \textrm{with} \:\: z = Hx
\end{equation}\\
\begin{equation}\label{eq:SGDUpdate}
\Delta h = - \rho \dfrac{\partial L}{\partial h} = - \rho \sum_{x} \dfrac{\partial l}{\partial z} x
\end{equation}\\
L’aggiornamento $\Delta H$ consiste in somme scalari di vettori $x$ proiettate sui filtri corrispondenti, quindi ereditano la maggior parte della struttura della covarianza dei dati in input. Di conseguenza se $x$ fa parte di un insieme di immagini naturali abbiamo che lo spettro della potenza è inversamente proporzionale a quello della frequenza \cite{field1987relations}, dunque lo step size per componenti a basse frequenze è molto più grande rispetto a componenti ad alte frequenze, questo può portare a problemi di convergenza e in casi rari di stabilità.\\
Un metodo per risolvere questo problema è l’uso di un algoritmo diverso, simile al gradient descent, l’algoritmo ADAM \cite{kingma2014adam}, la cui funzione di aggiornamento \ref{eq:adamUpdate} varia leggermente. Dove $m_{h}$ è una media costantemente aggiornata della derivata $\dfrac{\partial l}{\partial z}$ e $C_{h}$ è una matrice diagonale che rappresenta la stima della covarianza.\\
\begin{equation}\label{eq:adamUpdate}
\Delta h = - \rho C_{h}^{-\tfrac{1}{2}} m_{h}
\end{equation}\\
Essendo $C_{h}$ forzata ad essere diagonale non riesce a rappresentare adeguatamente la struttura della covarianza per immagini naturali.\\
Per risolvere questo problema Ballé propone una variante di questo algoritmo, SpectralADAM, dove invece di applicare l’algoritmo direttamente ad $h$, viene applicato alla sua Trasformata Discreta di Fourier con ingresso Reale (RDFT) riparametrizzando $h$ \ref{eq:sadamH}.\\
\begin{equation}\label{eq:sadamH}
H = F^{T} g \:\: \textrm{with} \:\: g = Fx
\end{equation}\\
Per ottimizzare $g$ usiamo la derivata \ref{eq:sadamGradient} ed applichiamo la regola di aggiornamento \ref{eq:adamUpdate}.\\
\begin{equation}\label{eq:sadamGradient}
\dfrac{\partial l}{\partial g} = F \dfrac{\partial L}{\partial h} = - \dfrac{\partial l}{\partial z} Fx
\end{equation}\\
Essendo \ref{eq:sadamH} lineare possiamo calcolare l’aggiornamento $h$ come descritto nell’equazione \ref{eq:sadamUpdate}, dove $m_{g}$ e $C_{g}$ sono delle medie costantemente aggiornate delle derivate \ref{eq:sadamGradient} \\
\begin{equation}\label{eq:sadamUpdate}
\delta h = F^{T} (-\rho C_{g}^{-\tfrac{1}{2}} m_{g}) = - \rho F^{T} C_{g}^{-\tfrac{1}{2}} Fm_{h}
\end{equation}\\
La stima della covaranza $ F^{T} C_{g} F$ deve essere diagonale nel dominio della trasformata di Fourier, non più nel dominio dei coefficienti dei filtri, e fino a quando $x$ è invariante per permutazioni, come tutti i dati spazio temporali, la base $F$ è garantito dia una buona approssimazione degli autovettori della vera struttura della covarianza dell’input $x$.\\
Dai risultati sperimentali della ricerca di Ballé, SADAM stabilizza e velocizza il processo di addestramento delle reti, inoltre essendo Ballé un ricercatore per Google, aveva già implementato il codice per queste soluzioni alternative all’interno della popolare libreria Tensorflow \cite{tensorflow2015-whitepaper}.\\
Alla luce di questi risultati ottenuti da Ballé ci chiediamo come mai negli anni successivi questo metodo non sia stato utilizzato per addestrare le nuove reti, ma si sia preferito continuare ad usare ADAM, come possiamo vedere nei lavori di Cheng et al. \cite{cheng2020learned} e Wang et al. \cite{wang2022neural}. Riteniamo quindi sarebbe interessante addestrare nuovamente queste reti utilizzando SADAM e valutare la differenza di prestazioni.


\section{Utilizzi in dispositivi mobili con Slim CAE}

Durante la ricerca delle fonti per la stesura di questo documento ci siamo imbattuti in una ricerca molto interessante del 2021 da parte di Yang et al. \cite{yang2021slimmable}, dove propongono di utilizzare delle SlimCAE, in modo da poter ridurre la potenza di calcolo necessaria per la compressione senza rinunciare a troppa qualità, tutto questo per poter rendere questa tecnologia fruibile anche su dispositivi con ridotta potenza di calcolo, come gli smartphone o più in generale dei dispositivi mobili.\\
Per rendere gli autoencoder, la cui struttura è stata descritta nel capitolo 2, degli slimmable autoencoder è necessario rendere i vali livelli della rete slimmable. Un livello, per essere slimmable, deve realizzare un’operazione valida rimuovendo parte dei parametri di quel livello.\\
Consideriamo gli slimAE come composti da $K$ sotto autoencoders, ognuno caratterizzato da due parametri \ref{eq:slimAEParameters} ed ognuno con la sua funzione di perdita \ref{eq:slimAELoss}. Di conseguenza i parametri di tutta la rete sono contenuti in un vettore di $K$ coppie di parametri e la funzione di perdita altro non è che la somma pesata, con dei pesi $w^{k}$, delle $K$ funzioni di perdita.\\
\begin{equation}\label{eq:slimAEParameters}
\psi^{k} = (\theta^{(k)},\phi^{(k)}) \:\: \in \:\: {(\theta^{(1)},\phi^{(1)}),\,…\, , (\theta^{(K)},\phi^{(K)})}
\end{equation}\\
\begin{equation}\label{eq:slimAELoss}
L(\Psi, \chi) = \sum_{k} w^{k} L^{k} (\theta^{k}, \phi^{k}; \chi)
\end{equation}\\
Per ottenere invece delle slimCAE dobbiamo rendere tutte le operazioni nei CAE non parametriche, riducibili o rimpiazzabili senza perdere efficienza. Nella rete proposta da Yang et al. la quantizzazione non è parametrica, i livelli convoluzionali sono stati implementati per essere riducibili, per GDN e IGND \cite{balle2018efficient} ci sono varie alternative che possono essere intercambiate ed infine vengono utilizzati dei modelli per l’entropia scambiabili, in questo modo ogni sotto CAE ha i suoi parametri $\nu^{k}$.\\
Possiamo vedere uno schema di funzionamento di una slimCAE nell’immagine \ref{fig:smallCAE}, osserviamo come andando ad incrementare la dimensione della rete possiamo comprimere più dettagli, realizzando quindi una codifica progressiva.\\
Dai risultati sperimentali ottenuti dal team di Yang et al. \cite{yang2021slimmable}, la rete da loro proposta ottiene risultati al pari del modello proposto da Ballé et al. \cite{minnen2018joint}, nonostante le dimensioni ridotte della rete e i tempi di codifica ridotti.\\
Ci chiediamo quindi come mai questo modello non abbia guadagnato popolarità data la sua applicabilità in situazioni più vicine alla realtà e per quale motivo le nuove reti non cerchino di realizzare anche delle versioni riducibili per permettere a sempre più persone di usufruire di tale tecnologia.\\
\begin{figure}[!h]
    \centering
    \includegraphics[width=0.4\textwidth]{Immagini/slimCAE.png}
    \caption{Diagramma funzionamento slimCAE, immagine presa dal documento \cite{yang2021slimmable}}
    \label{fig:smallCAE}
\end{figure}
    \chapter{Conclusioni}
Questo lavoro presenta l’attuale situazione per quanto riguarda la compressione di immagini con metodi tradizionali e che fanno uso di reti neurali. Dati i recenti avanzamenti nell’ambito di ricerca ci sentiamo di affermare che, alla luce degli esperimenti da noi eseguiti, le attuali reti, Ballé et al. \cite{minnen2018joint} e Cheng et al. \cite{cheng2020learned}, funzionano molto bene e potrebbero essere applicate nella vita di tutti i giorni, in quanto raggiungono prestazioni nettamente superiori rispetto ai metodi tradizionali più utilizzati, quali JPEG \cite{125072}, JPEG 2000 \cite{952804} e BPG \cite{BPGImageformat}, a discapito di una compressione più lenta eseguita su CPU, come possiamo vedere dai grafici \ref{fig:times07}, \ref{fig:times16}, \ref{fig:times21}, \ref{fig:times34}.\\
In casi specifici, dove è richiesta una qualità di compressione superiore invece, i metodi tradizionali e nello specifico VVC \cite{9503377}, rappresentano ancora l’unica alternativa sensata in quanto permettono di comprimere di più rispetto alle reti, garantendo una migliore qualità oggettiva e soggettiva, come possiamo osservare rispettivamente nei grafici \ref{fig:PSNRGraph} e \ref{fig:LPIPSGraph}.\\
L’ambito di ricerca della compressione con reti neurali rimane comunque molto promettente e con la crescente diponibilità di potenza di calcolo e di esempi di addestramento non potrà far altro che continuare a migliorare ulteriormente. Alcune migliorie, che potrebbero essere oggetto di future ricerche, potrebbero essere l’uso di SpectralADAM \cite{balle2018efficient} per l’addestramento delle reti di compressione e l’attenzione nello sviluppo di reti riducibili, come slimCAE \cite{yang2021slimmable} sviluppata dal team di Yang et al. nel 2021, per permettere l’uso di questi metodi anche su dispositivi con ridotta potenza di calcolo.\\
La naturale prosecuzione di questo lavoro sarebbe la valutazione di metodi che fanno uso di reti neurali per quanto riguarda la codifica video, valutando le prestazioni dei metodi tradizionali attualemente esistenti e compararli con metodi più recenti che fanno uso di intelligenza artificiale e metodi ibridi.\\

    \cleardoublepage

    \printbibliography[heading=bibintoc]
\end{document}
