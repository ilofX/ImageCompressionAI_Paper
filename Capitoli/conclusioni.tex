\chapter{Conclusioni}
Questo lavoro presenta l’attuale situazione per quanto riguarda la compressione di immagini con metodi tradizionali e che fanno uso di reti neurali. Dati i recenti avanzamenti nell’ambito di ricerca ci sentiamo di affermare che, alla luce degli esperimenti da noi eseguiti, le attuali reti, Ballé et al. \cite{minnen2018joint} e Cheng et al. \cite{cheng2020learned}, funzionano molto bene e potrebbero essere applicate nella vita di tutti i giorni, in quanto raggiungono prestazioni nettamente superiori rispetto ai metodi tradizionali più utilizzati, quali JPEG \cite{125072}, JPEG 2000 \cite{952804} e BPG \cite{BPGImageformat}, a discapito di una compressione più lenta eseguita su CPU, come possiamo vedere dai grafici \ref{fig:times07}, \ref{fig:times16}, \ref{fig:times21}, \ref{fig:times34}.\\
In casi specifici, dove è richiesta una qualità di compressione superiore invece, i metodi tradizionali e nello specifico VVC \cite{9503377}, rappresentano ancora l’unica alternativa sensata in quanto permettono di comprimere di più rispetto alle reti, garantendo una migliore qualità oggettiva e soggettiva, come possiamo osservare rispettivamente nei grafici \ref{fig:PSNRGraph} e \ref{fig:LPIPSGraph}.\\
L’ambito di ricerca della compressione con reti neurali rimane comunque molto promettente e con la crescente diponibilità di potenza di calcolo e di esempi di addestramento non potrà far altro che continuare a migliorare ulteriormente. Alcune migliorie, che potrebbero essere oggetto di future ricerche, potrebbero essere l’uso di SpectralADAM \cite{balle2018efficient} per l’addestramento delle reti di compressione e l’attenzione nello sviluppo di reti riducibili, come slimCAE \cite{yang2021slimmable} sviluppata dal team di Yang et al. nel 2021, per permettere l’uso di questi metodi anche su dispositivi con ridotta potenza di calcolo.\\
La naturale prosecuzione di questo lavoro sarebbe la valutazione di metodi che fanno uso di reti neurali per quanto riguarda la codifica video, valutando le prestazioni dei metodi tradizionali attualemente esistenti e compararli con metodi più recenti che fanno uso di intelligenza artificiale e metodi ibridi.\\
